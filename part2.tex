%----------------------------------------------------------------------------------------
%	PART
%----------------------------------------------------------------------------------------

\part{Rahmenbedingungen}

%----------------------------------------------------------------------------------------
%	CHAPTER 1
%----------------------------------------------------------------------------------------

\chapterimage{chapter_head_2.pdf} % Chapter heading image

\chapter{Rahmenbedingungen}

Bei der Realisierung des Projekts ist darauf zu achten, dass damit zahlreiche grundlegende und erweiterte Kompetenzen der Fachtheorie-Pflichtgegenstände abgedeckt werden sollen.

Dies betrifft voraussichtlich die im folgenden aufgeführten Gegenstände und Kompetenzen (in der Schulstufe 12 SoSe / 4. Jahrgang Sommersemester).
Angesichts des Projektumfangs ist eine Aufteilung auf mehrere Folgeprojekte durchaus möglich (ggf. auch in anderen Semestern bzw. als Diplomprojekt).


\chapter{Gegenstände}

\section{Softwareentwicklung}

\noindent {\color{red}Grundsätzlich: Verwendung von Frameworks; Dokumentation der Software-Architektur; Verwendung eines öffentlichen Repositories. Offene Schnittstellen vor allem in Hinsicht auf Folgeprojekte.}

\bereich{Anwendungsentwicklung}
\begin{itemize}[label={-}]
    \item Anwendungssysteme unter Verwendung von Nebenläufigkeit entwickeln
        {\color{red}(GUI-Front-End (Desktop/Web/mobile App))};
    \item[] einfache Schnittstellen zur Kommunikation zwischen Anwendungen entwerfen und implementieren
        {\color{red}(Schnittstelle Front-End/Back-End (WebService, WebSocket, REST, \dots))};
    \item[] Programme unter Berücksichtigung von Entwurfsmustern entwickeln
        {\color{red}(GUI Design-Patterns, Observer-Pattern, Command-Pattern, \dots)};
    \item[] Client-Server Anwendungen entwickeln
        {\color{red}(Peer-to-Peer, Client-Server; Front-End/Back-End; \dots)}.
    \item[\tiny\textsc{Lehrstoff}] Definition und Implementierung von Schnittstellen, Threading, Mehrschichtarchitektur, komplexere Entwurfsmuster, Umsetzen von Aufgabenstellungen aus den fachtheoretischen Gegenständen.
\end{itemize}


\section{Informationssysteme}

\bereich{Datenbankanwendungen}

\noindent {\color{red}Korrekte Evaluation / Prototypen für die Machbarkeit von konkreten Datenbank(management)systemen (relational vs. NoSQL) bzw. ORM-Systemen.}

\begin{itemize}[label={-}]
    \item standardisierte Datenbankschnittstellen installieren und konfigurieren, und damit aus gängigen Programmiersprachen mit einem Datenbanksystem kommunizieren
        {\color{red}(Korrekte Evaluation / Prototypen für die Machbarkeit von konkreten Datenbank(management)systemen (relational vs. NoSQL) bzw. ORM-Systemen.)};
    \item[] die Einsatzgebiete von datenbankseitiger Programmierung evaluieren und solche Anwendungen entwickeln
        {\color{red}(Zumindest Einsatz von \enquote{prepared Statements}, insbesondere zur Absicherung der Anwendung)};
    \item[] Anwendungen mit Datenanbindung entwickeln
        {\color{red}(Verwendung passender Technologien, Einsatz von Datenbank-Connectoren bzw. ORM-Systemen)};
    \item[] den Begriff „Transaktion“ erklären, die Voraussetzungen für eine korrekte Abarbeitung nennen sowie die Problematiken bei parallel auftretenden Transaktionen aufzeigen und diese in Fehlerklassen kategorisieren
        {\color{red}(Sinnvolle Absicherung bezüglich gleichzeitiger Datenbankzugriffe)}.
    \item[\tiny\textsc{Lehrstoff}] Aufbau, genormte Datenbank-Schnittstellen, Installation, Konfiguration, Vergleich von Schnittstellen; Einsatzgebiete von Stored Procedures, Trigger, Functions; Zugriff auf Daten aus gängigen Skript- und Programmiersprachen; Isolation Levels, Logs, ACID Kriterien.
\end{itemize}

\bereich{Informationssysteme und Contentmanagement}
\begin{itemize}[label={-}]
    \item die Anforderungen und Klassifizierungen von Informationssystemen angeben;
    \item[] marktgängige Contentmanagementsysteme installieren und konfigurieren
        {\color{red}(Projektwebsite mit Fortschrittsberichten, Projektdokumentation)};
    \item[] Content Management Systeme anwenden, entsprechende Inhalte entwickeln, sowie den entsprechenden Content Livecycle erläutern
        {\color{red}(Webseite für \enquote{langfristiges} Projekt (hinsichtlich von Folgeprojekten, Wartung, \dots); beinhaltet auch Support, Bugtracking, \dots)}.
    \item[\tiny\textsc{Lehrstoff}] Beurteilung marktgängiger Systeme; Installation und Konfiguration und Anwendung von Contentmanagementsystemen. 
\end{itemize}

\bereich{Informationsmanagement}
\begin{itemize}[label={-}]
    \item Informationsschnittstellen implementieren
        {\color{red}(Schnittstelle Roboter-Hardware / Back-End System (?))};
    \item[] die wichtigsten Aspekte in Geschäftsbeziehungen zwischen Unternehmen, Anbietern und Endverbrauchern beschreiben.
    \item[\tiny\textsc{Lehrstoff}] Betriebliche Informationssysteme: Informationsschnittstellen; Geschäftsprozesse: Beziehungen zwischen Anbietern und Endverbrauchern, Beziehungen zwischen Unternehmen.
\end{itemize} 


\section{Medientechnik}

\bereich{Multimedia}
\begin{itemize}[label={-}]
    \item die Grundregeln der Bildgestaltung in der Fotografie technisch korrekt umsetzen sowie Hard- und Software auftretende Probleme analysieren und lösen
        {\color{red}(Projektdokumentation, Benutzermanual, repräsentative Website)};
    \item[] eine projektbezogene Auswahl und Zusammenstellung des zu verwendenden Equipments treffen und mit den entsprechenden Dokumenten die gesamte Logistik (wann/was/wo/wer: Zeitplänen, Drehorte, Raumsetting, Team/Protagonisten) erstellen und anhand von Anforderungen geeignete Software und Techniken auswählen und damit multimediale Projekte umsetzen;
    \item[] Copyright von Bild- und Tonmaterial unterscheiden, derartige Materialien korrekt in Medienprojekten integrieren und mit Bild- und Audioprogrammen bearbeiten;
    \item[] die Schritte von der Idee zur Umsetzung (Präproduktion, Produktion, Postproduktion) erklären und damit eigene Projekte im Team umsetzen
        {\color{red}(Projektdokumentation, Benutzermanual, repräsentative Website)}.
    \item[\tiny\textsc{Lehrstoff}] Bildgestaltung, Filmgestaltung, Audiogestaltung; 3D-Programmierung; Copyright; Präproduktion, Produktion, Postproduktion.\lehrstoffrule
 
    \item eine Fotostrecke zu einem vorgegebenen Thema kreativ und technisch korrekt produzieren
        {\color{red}(Projektdokumentation, Benutzermanual, repräsentative Website)}.
    \item[\tiny\textsc{Lehrstoff}] Visual Storytelling.\lehrstoffrule
 
    \item visuelle Gestaltung in unterschiedlichen Medien umsetzen.
    \item[\tiny\textsc{Lehrstoff}] visuelle Grundelemente, Layout, Gestaltungsmittel.
\end{itemize}

\bereich{Datenbereitstellung, Web- und App-Technologien}
\begin{itemize}[label={-}]
    \item einfache Webapplikationen mit responsiven Designs für unterschiedliche Endgeräte mit client- und serverseitiger Programmierung erstellen
        {\color{red}(Web Front-End)};
    \item[] aus Webapplikationen auf heterogene Datenquellen zuzugreifen.
    \item[\tiny\textsc{Lehrstoff}] Webapplikationen, responsive Design, heterogene Datenquellen.\lehrstoffrule
  
    \item Multimedia-Anwendungen als mobile Applikationen entwickeln;
    \item[] Userinterface-Animationen in einer mobilen Applikation umsetzen
        {\color{red}(mobile App)}.
    \item[\tiny\textsc{Lehrstoff}] Audio- und Video-APIs, GUI-Entwicklung.
\end{itemize}

\bereich{3D-Modellierung, Animation, Interaktion und Simulation}
\begin{itemize}[label={-}]
    \item grundlegende 3D-Postproduktionseffekte erklären und anwenden;
    \item[] gerenderte Bilder in einem genau definierten Format produzieren;
    \item[] eigene Kameras erstellen;
    \item[] eigene 3D-Objekte für andere Anwendungen bereitstellen
        {\color{red}(Simulation in Desktop/mobile App)}.
    \item[] selbst produzierte Grafiken aus einer 3D-Produktion mit Farbmanagement adaptieren und in verschiedene Medien integrieren
        {\color{red}(Projektdokumentation, Benutzermanual, repräsentative Website)}.
    \item[\tiny\textsc{Lehrstoff}] 3D-Postproduktion, Bildformate, Kameras, Datenexport; Farbmanagement, Datenbereitstellung.\lehrstoffrule

    \item mit nichtlinearen Animationstechniken komplexe Szenen erstellen
        {\color{red}(Simulation: Roboter beschleunigt, bremst, \dots)};
    \item[] die Grundlagen von Partikelsystemen erklären
        {\color{red}(Simulation: diverse Hindernisse (Gas), brennender Roboter, \dots)}.
    \item[\tiny\textsc{Lehrstoff}] Nichtlineare Animationen, Partikelsysteme.
\end{itemize}

\bereich{Game Development und Simulation}
\begin{itemize}[label={-}]
    \item Grafikeffekte und Materialien in Game Engines erklären und modifizieren
        {\color{red}(Roboter und Spielfeld; Hindernisse)};
    \item[] Lichter und Beleuchtung in Echtzeitanwendungen erklären und modifizieren
        {\color{red}(Spielfeldbeleuchtung; Lichter; Laser; \dots)};
    \item[] Scripts zur 2D/3D-Simulation erstellen
        {\color{red}(Steuerung der Spielelogik)};
    \item[] Animation von Spielobjekten erstellen
        {\color{red}(Roboter bewegt sich, Elemente am Spielbrett beweglich (z.B. Förderband))};
    \item[] Animation und Gamelogik mittels Programmierung verbinden
        {\color{red}(Implementierung \& Testen)};
    \item[] Co-Routines bzw. Multithreading in Game Scripts einsetzen
        {\color{red}(Multithreading, Roboter mit eigenen Prozessen, \dots)}.
    \item[\tiny\textsc{Lehrstoff}] Physik-Engines, Echtzeit-Grafik, Co-Routines und Threading, Animation.
\end{itemize}

\bereich{Ethische Aspekte, Rechtliche Grundlagen und Gesellschaftliche Auswirkungen der Informationstechnologie -- Basiswissen}
\begin{itemize}[label={-}]
    \item die Interaktion zwischen Informationstechnik, Gesellschaft und Politik beschreiben und erläutern;
    \item[] Kommunikationsfreiheit und Kommunikationsrechte beschreiben und erklären;
    \item[] die Entwicklung und Arbeitsweisen der Datenwirtschaft beschreiben und erläutern;
    \item[] Grundlegende Begriffe der Medienethik nennen und erklären;
    \item[] den Verantwortungsbegriff differenziert beschreiben und erklären;
    \item[] informationsethische und rechtliche Verantwortung im Hinblick auf ihre Erzeugnisse und ihr informationstechnisches Handeln in Beziehung setzen.
    \item[\tiny\textsc{Lehrstoff}] Medienethik.
\end{itemize}


\section{Systemtechnik}

\bereich{Industrielle Informationstechnik}
\begin{itemize}[label={-}]
    \item Sensordaten aufnehmen, aufarbeiten und über gängige industrielle Kommunikationssysteme zur Verfügung stellen
        {\color{red}(Roboter-Hardware: Feststellung des Spielfeldes, Stromverbrauch, Hindernisse, \dots)}.
    \item[\tiny\textsc{Lehrstoff}] Bussysteme, hardwarebasierte Datenaufnahme und -verarbeitung.\lehrstoffrule
 
    \item externe Signale in Roboterprogrammen verarbeiten und Signale an externe Systeme weitergeben
        {\color{red}(Sensorik im Roboter und am Spielfeld; Kommunikation zwischen Roboter und Spielfeld)};
    \item[] Kommunikations-Schnittstellen auswählen, an ein Handhabungssystem koppeln und in Betrieb nehmen
        {\color{red}(Eventuell: Steuerung/Programmierung autonomer Komponenten am Spielfeld)}.
    \item[\tiny\textsc{Lehrstoff}] Erweiterte Teach-In-Programmierung, Kontrollstrukturen, Variablen, Programmstrukturierung, HMI-Anbindung.\lehrstoffrule
 
    \item verschiedene Bewegungsarten des Handhabungssystems anwendungsgerecht kombinieren
        {\color{red}(Eventuell: Steuerung/Programmierung autonomer Komponenten am Spielfeld)};
    \item[] Programme ausführen und mit Hilfe von Teach-In-Techniken und Offlineprogrammierung erstellen und systematisch testen
        {\color{red}(Eventuell: Steuerung/Programmierung autonomer Komponenten am Spielfeld)}.
    \item[\tiny\textsc{Lehrstoff}] Erweiterte Teach-In-Programmierung, Kontrollstrukturen, Variablen, Programmstrukturierung, HMI-Anbindung.
\end{itemize}

\bereich{Systemintegration und Infrastruktur}

\noindent {\color{red}(Ohne passenden Deskriptor im Semester: Realisierung der Back-End Komponenten in Containern.)}
\begin{itemize}[label={-}]
    \item geeignete unternehmensweite Kommunikationsmittel beschreiben, vergleichen, installieren und betreiben
        {\color{red}(Synchrone/Asynchrone Kommunikation zwischen den Spielern (Text/Sprache))};
    \item[] die für Voice over IP benötigten Standards und Protokolle beschreiben sowie geeignete Implementierungen installieren und betreiben
        {\color{red}(Synchrone/Asynchrone Kommunikation zwischen den Spielern (Sprache))}.
    \item[\tiny\textsc{Lehrstoff}] Kommunikationsdienste, Voice over IP.\lehrstoffrule
  
    \item unterschiedliche Zugriffskontrollmechanismen für Systeme vergleichen und Systeme damit geeignet absichern
        {\color{red}(Authentifizierung in Front-End und Back-End; Absicherung der Roboter-Hardware (keine \enquote{anonyme} Programmierung))};
    \item[] verschiedene Firewall-Typen beschreiben und geeignete Systeme in einer Topologie an geeigneter Stelle einsetzen, konfigurieren und betreiben
        {\color{red}(Back-End geeignet absichern)}.
    \item[\tiny\textsc{Lehrstoff}] Access Control, Firewall.
\end{itemize}

\bereich{Dezentrale Systeme}
\begin{itemize}[label={-}]
    \item in dokumentenbasierten dezentralen Systemen eingesetzte offene Dokumentenformate und Auszeichnungssprachen zur Realisierung solcher Systeme einsetzen
        {\color{red}(Datenübertragung zwischen Front-End(s) und Back-End)}.
    \item[\tiny\textsc{Lehrstoff}] Enterprise-Frameworks, Dokumentenorientierte Middleware Systeme, Einsatz von NoSQL/REST/JSON.\lehrstoffrule
  
    \item ein dokumentorientiertes Middleware Systemen konzipieren und implementieren
        {\color{red}(vermutlich eher Nachrichtenorientiert? Eventuell für Leaderboard, Meisterschaften, \dots)}.
    \item[\tiny\textsc{Lehrstoff}] Enterprise-Frameworks, Dokumentenorientierte Middleware Systeme.\lehrstoffrule

    \item ein dezentrales System mit Hilfe von webbasierten Frameworks umsetzen
        {\color{red}(Peer-to-Peer bzw. Client-Server Anwendung)}.
    \item[\tiny\textsc{Lehrstoff}] Enterprise-Frameworks, Einsatz von NoSQL/REST/JSON, Konfiguration und Datenübertragung mittels gängigen Dokumentenformaten.\lehrstoffrule
  
    \item Programmiertechniken in verteilten Systemen zur Realisierung von entfernten Prozeduren, Methoden und Objekten anwenden
        {\color{red}(Peer-to-Peer bzw. Client-Server Anwendung)}.
    \item[\tiny\textsc{Lehrstoff}] Enterprise-Frameworks, Einsatz von entfernten Prozeduren und verteilten Objekten, Konfiguration und Datenübertragung mittels gängigen Dokumentenformaten.
\end{itemize}

\bereich{Ethische Aspekte, Rechtliche Grundlagen und Gesellschaftliche Auswirkungen der Informationstechnologie -- Basiswissen}
\begin{itemize}[label={-}]
    \item die Interaktion zwischen Informationstechnik, Gesellschaft und Politik beschreiben und erläutern;
    \item[] Kommunikationsfreiheit und Kommunikationsrechte beschreiben und erklären;
    \item[] die Entwicklung und Arbeitsweisen der Datenwirtschaft beschreiben und erläutern;
    \item[] Grundlegende Begriffe der Datenethik nennen und erklären;
    \item[] den Verantwortungsbegriff differenziert beschreiben und erklären;
    \item[] informationsethische und rechtliche Verantwortung im Hinblick auf ihre Erzeugnisse und ihr informationstechnisches Handeln in Beziehung setzen.
    \item[\tiny\textsc{Lehrstoff}] Datenethik.
\end{itemize}


\section{Informationstechnische Projekte}

\bereich{Durchführung informationstechnischer Projekte}
\begin{itemize}[label={-}]
    \item Teile des Lastenheft und einer Evaluation selbständig für ein Projekt erstellen
        {\color{red}(Lastenheft anhand der Spielregeln vervollständigen)};
    \item[] Teile einer Evaluation selbständig für ein Projekt erstellen
        {\color{red}(Prototypen für verschiedene Technologien: DB (relational vs. NoSQL, welches ORM) -- Sprache FE/BE (Java/Kotlin; Objective-C/Swift; Java/Python/Node.js/\dots) -- Welcher Controller für Hardware? Welche Sensoren? -- Frameworks für Frontends (Web/App/Desktop)? -- Engine für Simulation (3D, Physik, \dots))};
    \item[] Teile der Useranforderungen entsprechend der gewählten agilen Projektmanagementmethode erstellen
        {\color{red}(Siehe Lastenheft; anhand der konkreten Spielregeln Erstellung von Arbeitspaketen unter Berücksichtigung der daran hängenden Kompetenzen der Fachtheoretischen Gegenstände)};
    \item[] eine korrekte und zeitnahe Aufzeichnung der eigenen Arbeitspakete bzw. Useranforderungen erstellen
        {\color{red}(Siehe Lastenheft; anhand der konkreten Spielregeln Erstellung von Arbeitspaketen unter Berücksichtigung der daran hängenden Kompetenzen der Fachtheoretischen Gegenstände)};
    \item[] mittels Projektpräsentation die eigenen Aufgaben sowie Inhalte und Status des Projektes überzeugend darstellen; einen aktuellen Statusbericht des Projektes erstellen
        {\color{red}(Präsentation auch beim Tag der offenen Tür; Präsentation an \enquote{Folgeklassen} (auch für Folgeprojekte))}.
    \item[\tiny\textsc{Lehrstoff}] Planung und Realisierung informationstechnischer Projekte unter Wahrnehmung typischer Rollenbilder und unter Berücksichtigung von Themenbereichen der technischen Pflichtgegenstände mittels agilen Projektmanagementmethoden.
\end{itemize}

\bereich{Projektmanagement}
\begin{itemize}[label={-}]
    \item Grundlagen der Aufwandschätzung erklären und auf Basis eines Projektes eine passende Variante auswählen
        {\color{red}(Aufwandsschätzung für einzelne Arbeitspakete unter Berücksichtigung der zu erreichenden Kompetenzen)};
    \item[] wesentliche Projektrisiken und Chancen erkennen und geeignete Maßnahmen dafür vorsehen
        {\color{red}(Risikomanagement)};
    \item[] Kreativitätsmethoden von Kreativitätstechniken unterscheiden und den kreativen Prozess beschreiben;
    \item[] Projektkrise und die Begriffe erklären, sowie unterschiedliche Ursachen und Indikatoren für Krisen beschreiben.
    \item[\tiny\textsc{Lehrstoff}] Projektplanung, Projektrisiko, Projektdokumentation, Multiprojektmanagement, Kreativität einsetzen, Krisen analysieren.
\end{itemize}


\section{IT-Sicherheit -- MEDT}

\bereich{Secure Web- \& App-Development}
\begin{itemize}[label={-}]
    \item Web-Anwendungen und mobile Applikationen absichern
        {\color{red}(Absicherung Front-End (und Back End?))};
    \item[] rechtliche Grundlagen erklären und bei der Entwicklung berücksichtigen
        {\color{red}(Datenschutz, Zertifikate, Verschlüsselung)}.
    \item[\tiny\textsc{Lehrstoff}] OWASP Top 10, Web-Application-Firewalls, Datenschutz, Urheberrecht, Zertifikate, Verschlüsselung.
\end{itemize}


\section{IT-Sicherheit -- SYT}

\bereich{IT-Security}
\begin{itemize}[label={-}]
    \item Prinzipien von Penetrationstests erläutern und im rechtlichen Rahmen entsprechende Angriffe durchführen
        {\color{red}(Testen von Back-End und Peer-to-Peer Front-Ends)}.
    \item[\tiny\textsc{Lehrstoff}] Permission-to-Attack, Responsible Disclosure, Data-Privacy.\lehrstoffrule
 
    \item Systeme und deren Schnittstellen gegen Angriffe schützen und härten
        {\color{red}(Absicherung von Front End und Back End)}.
    \item[\tiny\textsc{Lehrstoff}] Buffer-Overflows, Mandatory-Access-Control, Trusted-Platform; IDS, IPS; Datenintegrität; Fuzzing.
\end{itemize}


\section{Artificial Intelligence -- MEDT}

\bereich{Grundlagen der Künstlichen Intelligenz}
\begin{itemize}[label={-}]
    \item Algorithmen der KI erkennen und erstellen
        {\color{red}(Möglichkeiten für KI-gesteuerte Roboter evaluieren)};
    \item[] Systeme zum autonomen Agieren mit Benutzern entwickeln
        {\color{red}(\enquote{Support-Bot} auf Website?)};
    \item[] einfache prozedurale und KI basierte Generierung von Medieninhalten erstellen
        {\color{red}(Automatische Erstellung von: Spielfeldern; Robotern)}.
    \item[\tiny\textsc{Lehrstoff}] KI in Game Engines, Finite State Machines, Suchalgorithmen und deren Anwendungen, Grundlagen Machine Learning und Klassifikation, prozedurale Content-Erstellung.
\end{itemize}


\section{Artificial Intelligence -- SYT}

\bereich{Data Science}
\begin{itemize}[label={-}]
    \item die Qualität von Daten und Algorithmen anhand von Output-Daten überprüfen
        {\color{red}(Festsetzung geeigneter Kriterien für die nachträgliche Auswertung des Projektes (insbesondere Auswertung von Spielern und Spielen); ggf. Einführung eines entsprechenden Scoring-Systems)}.
    \item[\tiny\textsc{Lehrstoff}] Ablauf von Datenanalyse oder maschinellen Lernprozessen durch Exploration; Daten in Trainings- und Testdatensätze aufteilen; Kenntnis von In-sample Schätzungen und Prädiktionen und Out-of-sample Schätzungen und Prädiktionen; Qualitätsprüfung von Algorithmen (z.B. mittels K-facher Kreuz-Validierung); Probleme der Modellanpassung (z.B. Overfitting, Underfitting).\lehrstoffrule
 
    \item gelernte Methoden im Rahmen aktueller Anwendungsgebiete umsetzen
        {\color{red}(Auswertung ähnlicher Daten (Auswertung von Wettbewerben, Spielern, \dots) z.B. auch bei Schach oder Go (weil vielfältige Daten verfügbar))}.
    \item[\tiny\textsc{Lehrstoff}] Umsetzung von Lernprozessen mithilfe von Algorithmen mit dem Ziel der Musterkennung: Kenntnis von Konzepten von Distanzmaßen (z.B. lineare Diskriminanzanalyse, Cluster Analyse (k-nearest neighbors, model based Clustering), Support Vector Machines); Fallstudie: Anwendung von Methoden der Exploration, Modellierung und Qualitätsprüfung anhand eines realen Datensatzes oder einer realen Fallstudie.
\end{itemize}

